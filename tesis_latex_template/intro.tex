%% ---------------------------------------------------------------------------
%% intro.tex
%%
%% Introduction
%%
%% $Id: intro.tex 1477 2010-07-28 21:34:43Z palvarado $
%% ---------------------------------------------------------------------------

\chapter{Introducción}
\label{chp:intro}

Hoy en día es cada vez más común el tema de las energías limpias, dentro de estas: eólica y solar. La instalación de suministros con paneles fotovoltaicos ha llegado a ser una tendencia en Costa Rica, estos son sistemas de autoconsumo de energía, y a su vez se utilizan para vender energía a otras empresas. 

Un sistema de abastecimiento de energía solar, requiere de paneles solares, acumuladores de energía, inversores (conversión de corriente continua en corriente alterna) y reguladores, sin embargo actualmente las redes de suministros no cuentan con un sistema que les regule la tensión que se necesita para ubicarse en el punto de operación de potencia máxima, esto debido a que la corriente y la tensión del panel están variando constantemente respecto a la temperatura e irradiancia del medio en el que se encuentra, de manera que si la tensión varia, la potencia asociada a esa tensión también varía, lo que se busca es un punto de tensión donde se obtenga la máxima potencia.
 
Debido a la importancia de la eficiencia energética en el campo de la electrónica, se desarrolló parte de un sistema en donde se puede aprovechar de una mejor manera la energía, este tema es de suma importancia en la producción de energía, principalmente en los paneles fotovoltaicos, se debe aprovechar las mejores condiciones ambientales y poder acoplar el sistema para una máxima producción de energía, el desarrollo de este proyecto se basa en aumentar la eficiencia del sistema completo para un panel previamente escogido, se utilizará un panel modelo KS10T de la empresa KYOCERA SOLAR, para esto se realiza una realimentación con un dispositivo que regula la tensión máxima que debe tener el panel. Anteriormente se realizó un estimador de parámetros por parte de Clevis Lozano estudiante del Instituto Tecnológico de Costa Rica, sin embargo este recibe en la entrada cuantificaciones lineales para calcular los parámetros requeridos, la curva característica de una celda solar no tiene un comportamiento lineal, si se requiere estimar parámetros a partir de la corriente y tensión de este, se deben linealizar-normalizar las entradas y desnormalizar-deslinealizar las salidas, para el modelo del panel se analizaron cuatro tipos de configuraciones desde la más simple a la más compleja, para este proceso se desarrolló un sistema de verificación en un programa de alto nivel (utilizando Python) para poder comparar y tener una referencia para la descripción en hardware de bajo nivel (verilog). 

El proyecto Linealizador-Normalizador se realizó en el Instituto Tecnológico de Costa Rica, Escuela de Ingeniería Electrónica, con el coordinador del SESLab Dr. Carlos Meza, y el coordinador del DCILab Dr. Alfonso Chacón estos laboratorios se encargan de presentar propuestas de sistemas electrónicos de gran utilidad para  para el desarrollo tecnológico y sostenibilidad, de manera que los recursos sean aprovechados de la mejor forma, brindan soluciones innovadoras con energías limpias, enfatizándose en el uso de paneles solares, motores eléctricos, circuitos integrados, diseño digital, entre otros. 


\section{Objetivos y estructura del documento}

\index{objetivos}

El objetivo general para este proyecto se basa en desarrollar una unidad de linealización y normalización para un estimador de parámetros Corriente-Tensión de un panel fotovoltaico, para esto se creo un circuito que linealice el modelo del panel fotovoltaico por medio de una operación logarítmica con parámetros de entrada Tensión-Corriente, generando parámetros lineales de salida ‘y’ y ’z’, estos parámetros de salida poseen formato IEEE 754 (punto flotante), por lo que se implemento un circuito que convierte estos parámetros de punto flotante a punto fijo. Estos parámetros se requieren en las entradas del estimador por que se debió realizar una unidad de normalización para crear una interface entre ambos circuitos y poder acoplar las salidas del linealizador con las entradas del estimador.

 \begin{figure}[H]
  \centering
    \includegraphics[scale=0.6]{./DiagramaSOL.png}
    \rule{35em}{0.5pt}
  \caption[Diagrama de solución]{Diagrama de solución}
  \label{fig:DSOL}
\end{figure}





